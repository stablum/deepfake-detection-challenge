\documentclass{article}
\usepackage{graphicx}
\usepackage{hyperref}
\usepackage{verbatim}
\graphicspath{ {./images/} }

\title{Deepfake detection challenge - a research journal}
\author{Francesco Stablum}

\date{\today}

\begin{document}

\maketitle

\section{Motivation}

If there is something that I miss in my education it's probably
machine learning applied on 3D rasters and temporal data.
The deepfake challenge seems to be the perfect opportunity
to dive deep in methods that I haven't tried before.
Specifically, analysis of video data may employ techniques from both domains
as videos can be interpreted as 3D grids, as well as a sequence of datapoints that
is subject to causal evolution. The first interpretation would suggest using
convolutional neural network with a 3D kernel; the latter suggests recurrent neural networks
and LSTMs.


\section{23/01/2020 - first attempts with a 3D convolutional neural network}

\subsection{The dataset}
The dataset comprises 400 videos that have about 300 frames each; sometimes
299 - with 3 colors channels.
The resolution is 1920x1080. Sometimes transposed (vertical smartphone videos).

Even if the datapoints are effectively quite similar, they still require some operations
to make them have the same dimensionality. Moreover, the sheer size of the uncompressed
video suggests some preliminary scaling down of the input data.

\subsection{Tackling the huge dimensionality of the datapoints}

Every datapoint, loaded trough the library \texttt{skvideo}, has an uncompressed
size of 1.8 GB. In these conditions it seems an obvious choice to load one datapoint at once.
In order to have faster epochs, a slice of 15 frames is selected from each datapoint
scaling down the datapoint dimensionality to 93 MB.
The starting frame number of the frame slice will be chosen randomly. This will allow
all the frames of a video to be eventually considered trough all the epochs.

\subsection{Structure of the first model}

\begin{figure}[h]
\caption{First 3d-convolutional neural network}
\centering
\includegraphics[width=0.5\textwidth]{first_model.png}
%\includegraphics{first_model.png}
\end{figure}

The convolutional layers have been set with ELU activation functions.
The reason for this is to avoid dead units that emerge with ReLU activation functions
caused by 0 gradients.
For the dense layers the sigmoid activation function has been chosen.
Different activation functions could be also be investigated in the future,
such as ReLU and tanh, respectively for the convolutional and dense layers.
Also, not much tought has been given to regularization. But, since the network
seem to not fit on the training set, that's a problem for later.

The amount of parameters for this model is 22648.

\begin{figure}[h]
\caption{parameter count for each layer}
\centering
\verbatiminput{first_model_summary.txt}
\end{figure}

\subsection{Considerations on using kernels that operate trough time}

What I noticed by watching the deepfake videos from the dataset is that some
of the videos were showing very scattered application of the face alteration.
Such abrupt changes are one of the first applications of convolutions for
image processing, such as edge detectors. Such characteristic would be 
detected here w.r.t. time.

For future reference, exclusively time-based convolution layers might be used
in a model - i.e. convolutional layers with kernel sizes that are 1 in both spatial
axes but arbitrary in the time dimension.

Also, factorized convolution could be used in a future model, as described in \cite{Burlacu}.

\subsection{Network output}

Since the output of the network is a 2-node softmax,
the first node is being used to learn a probability quantity of the input datapoint 
of being a deepfake. The second node, since it's softmax, will always output
1 minus the output of the first node.

The target vectors are hence $(1,0)$ for a deepfake and $(0,1)$ for a real video.

\subsection{Training}
Since training takes a considerable amount of time (about one hour for a training set
sweep), the number of training epochs is chosen to be as small as 100.
SGD has been chosen as optimization method. Adam/Nadam/Amsgrad/ND-Adam might be considered
as future improvement.

\subsection{Evaluation measures}

Currently two poorly-named (by myself) evaluation measures are being employed: unquantized average error,
and quantized average error.
Considering uniquely the value of the first output node, an unquantized error is the
difference between the expected value - 1 for deepfake, 0 for real - and the
emitted probability from the network. The unquantized average error
is simply the average value across the dataset.

The quantized error is, conversely, a rounding to the nearest integer of the unquantized error.
In a typical classification model a decision has to be made on how to use the output
of a neural network. Hence, the integer-rounding of the output represents the
decision on interpreting the binary classification model output either of one or the other category.
Hence, the quantized error is either 1 or 0 whenever that decision was correct or wrong.
The quantized average error is hence the average of the individual quantized errors
after a sweep of a validation/testing set.

In order to have a better overview on how the model is performing with regard to
the class unbalance problem, ratios of correct and wrong predictions disaggregated 
per class are being kept.

\subsection{Class unbalance problem}

The first model is not performing well. The output category is always the most frequent 
one, which is the deepfake videos, amounting to 323, while the real videos are only 77.
This class unbalance problem could be tackled by submitting more frequently
the real videos to the training. The ratio of fake/real is 4.2, hence one possible
future improvement could be to submit about 4 different slices of the real video to 
the fitting function for every slice of fake videos.

\subsection{Future improvements}
\begin{itemize}
\item localize the face and follow it to remove useless background
\item log number of parameters
\item log average loss
\item factorized convolutions
\item bottleneck layers
\item skip connections
\item different update methods
\item LSTMs/RNN
\item checkpoints to resume training
\end{itemize}

\section{25/01/2020 : second model}

\subsection{Decomposed kernels}

It's well known that a symmetric matrix can be decomposed in two vector-sized kernels.
Subsequent convolution with the two (in the case of 2D images) separated kernels
result into the same convolution as the original kernel.
An example is shown in \cite{Sobel}.

In theory it's possible to learn symmetric matrices (or tensors) by separation
in vectors, each for every dimension.

This is going to be the focus of the second model: every 3D convolution is
replaced by three 3D convolutions that have kernel dimension set at 1 except for the
dimension in which the convolution is going to operate.

This way, every ``block'' of convolutions is going to have 30 parameters instead of 246.

The 'stride' subsampling that was characteristic of every convolutional layer
in the first model is being set only at the third layer in each block of
``decomposed'' convolutional layers.

\subsection{Excessive computation of separated convolution layers}

Unfortunately this network, even if it has a lower number of parameters,
its still very computationally expensive.
This is possibly due to the triplication of sweeps on data that has the input 
dimensionality.

To tackle this problem, a subsampling maxpool layer has been added as input layer to
have a preliminary dimensionality reduction.
This follows an observation of genetic algorithm-evolved convolutional neural networks
that were optimized also by running time.
Many of the best-fit genotypes selected by the genetic algorithm
were putting a maxpool layer
on the input with suprising results also in accuracy of the network.

The resulting network has a much reduced amount of trainable parameters: only 1660.

\begin{figure}[h]
\caption{Second model: separated convolutions}
\centering
\includegraphics[width=0.5\textwidth]{second_model.png}
%\includegraphics{first_model.png}
\end{figure}

\begin{figure}[h]
\caption{Second model: parameter count for each layer}
\centering
\verbatiminput{second_model_summary.txt}
\end{figure}

\subsection{Excessive loading and pre-processing times}

Moreover, loading a single video in tensor form takes about 7 seconds.
This is not different even using a ramdisk, where all the videos are stored in
memory instead of on the hard drive.

This issue requires a different strategy. Under the current model configuration,
just 15 frames are being used over all 300 frames of each video. Considering the
high loading time of a single video this is quite a waste.
A better approach could be to load and process multiple 15-frames slices after a video
has been loaded.

This will cause a stronger ``learning intensity'' per epoch, as multiple calls to
the fit function will happen for every video.

Memory limits need to be taken into consideration with these changes.
A 15-frames uncompressed slice requires 93.3 MB. A conservative memory allocation
can be 2GB, hence about 20 slices can be loaded from a video.

Such changes come very handy also to tackle the class unbalance problem, which can
be dealt by submitting 5 slices for every fake video and 21 slices for every real
video, which will amount to the previously reported 4.2 ratio of fake videos / real videos.

By submitting 5 slices for every fake video and 21 slices for every real one, one ``new'' 
epoch will have processed 3232. Hence, it would be equivalent as $8.08$ epochs in the 
first model.

To keep a fair comparison with the first model, the amount of videos loaded at every epoch
will be one eight, i.e. 50 videos.

\subsection{Changes in activation functions and optimizing method}

The fully connected layers have been added the $\tanh$ activation function.
The co-domain of the function between -1 and +1 has possibly some advantages.
Vanishing gradient might still be an issue, hence ELU should be considered in the future.

Adam has been used as optimization method, due to using momentum.

\subsection{A note on the platforms used}

\emph{MLFlow} it's a python library employed to track all the experiments, artifact archive,
parameter and metric logging. It provides a convenient user interface to navigate
plots of evolutions of the metrics trough time.

\emph{Floyd Hub} is a remote service that provides easy deployment of runs 
on powerful
hardware such as nodes with NVidia K80 GPUs. It's a rental, so it comes with a price.


\subsection{Second model - an additional larger variant}

As a way to make proper use of the newly-rented GPUs from floydhub,
a larger version of the aforementioned model has been trained.
Among the differences: the number of convolutional blocks has been
increased from 3 to 7. This was made possible by the removal of the strides in the 
ending layer of each convolutional block. This reduced the dimensionality reduction
allowing more convolutional blocks (and max pool layers).
Moreover, the full initial dimensionality has been given to the convolutional blocks by
removing the initial maxpool layer.
Also, the first dense layer has been increased from 20 to 50. 
These changes resulted in a much higher trainable parameter count: $47,075$

\begin{figure}[h]
\caption{Second model: separated convolutions - larger variant}
\centering
\includegraphics[width=0.3\textwidth]{second_model_larger.png}
\end{figure}

\begin{figure}[h]
\caption{Second model - larger variant: parameter count for each layer}
\centering
\scriptsize
\verbatiminput{second_model_larger_summary.txt}
\end{figure}

\subsection{Performances of the second model and the larger variant}

Both models failed again to even hint at the learning some differences between the
two categories, having mean accuracy oscillating around 0.5, basically a coin toss.

\begin{figure}[h]
\caption{Second model: accuracy - vanilla and larger variant}
\centering
\includegraphics[width=1.3\textwidth]{second_model_mean_accuracy.png}
\end{figure}

\subsection{Reflecting on mistakes and future directions}

First of all, in the second model the advice given by \cite{Burlacu}
of \emph{not} using separated convolutions on the first layers was mistakenly not
considered. Moreover, having separated convolutions does not mean reduced computation
by the triplication of the number of sweeps. Especially when the kernel size
is kept as low as 3. It may make more sense if the kernel size would be much higher.

Another huge issue is that $90\%$ of the video is useless for the task of deepfake detection.
It may be more useful to identify a smaller window where the face is located
and concentrate the learning only on that sub-tensor. That would result in faster
learning times,hence more dataset sweeps (epochs) and inferior signal/noise ratio. A library such as \cite{pyface} can be used for the task.

Also, every fit call is currently processing one single datapoint, minibatch learning
needs to be seriously considered to take full advantage of the GPU.

Industry practices suggest using an incresing number of filters as the network gets deeper.




\bibliography{bibliography}{}
\bibliographystyle{plain}

\end{document}
